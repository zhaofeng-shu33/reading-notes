\documentclass{ctexart}
\usepackage{enumitem}
\usepackage[symbol]{footmisc}
\usepackage{sidenotes}
\title{汉语的优越性}
\author{王佩伦}
\date{2013年7月5日}
\begin{document}
\maketitle

今日填报志愿,我依旧不死心地填报了北外的提前批。能够学习德法等用途广泛的语言固然很好,
万一被调剂到了冰岛语等极冷僻专业我也会心悦诚服。于是想起了曾经看过的一篇让人心生感触的文章,特来分享~聊以慰藉你那颗因文言而纠结的心。

P.S.“聊以慰藉”一词的释义为:勉强、暂且用来宽慰自己,形容很勉强,有些自己敷衍自己的意思。
然而鲁迅也有过“聊以慰藉那在寂寞里奔驰的猛士,使他不惮于前驱”的话,\sidenote{《呐喊》自序 }
所以这慰藉的究竟是自己还是他人呢?抑或是鲁迅又一次的独创?

文章如下。

汉语无疑对外国人很难,但客观的说它也许还不是世界上最难的语言.比如说阿拉伯语,希腊语,俄语,德语.....?

世界上最难的语言之一,汉语,许多人是这样认为的,尤其是对汉语不了解的外国人.
汉语是三位一体的语言,音形意合而为一,和其他语言不同的是除了表“音”,同时还体“形”,会“意”。
最大的难点就在于音形意的结合。
很多时候我们看见一个字,知道意思,但是不会读,所以出现很多白字先生;\sidenote{如风不会表意}
也有的是会读,也知道意思,但就是不会写;\sidenote{英语亦然}
还有就是经常会忘记一个常用字怎么写。
相信大家都有体会,反正我本人深有体会,我从上中学开始就喜欢读不同的书,一直到大学快毕业,电脑出现后才读的少些,我相信自己比一般人看的书都要多。即使这样,我也会碰到不会读的字,不会写的就更多了。尤其是我在国外呆了多年之后,提笔忘字的现象常常出现。
但汉语最神奇的一点就是不会读也不会写的字,但意思很少有不知道的。

但与此同时,再和欧洲人交往的过程中,我也体会到了汉语,我也体会到了汉语是世界上最简单的语言。汉语的常用字一般就两到三千个左右,我有一个习惯,就是无论在什么地方,包括在国外的7年,汉语词典从不离身,就是大家最长用的那本小字典。我看书的时候遇到不知道读音的字,一般都会查字典,这么多年,几乎没有碰到过这本从小学开始用的字典有查不到的字。
相信学外语的人都碰到过这种情况,拿一本巨大的词典,可查不到的单词比比皆是。\sidenote{词不等于字,现代汉语词典查不到的词也比比皆是。}

前一段时间我教一个外国人学汉语,他的学习速度之快,令我大为惊讶。几堂课之后他就能简单的说一些,他之前已经学习了汉语拼音。我认识到汉语既是世界上最难学的语言,也是最容易掌握的语言。我一直认为学外语最终要得就是说,因为绝大多数情况下我们用的是“说”。所以我教外国人汉语是从说开始的,我从他最常用,最想说的话开始。我说,然后他自己用拼音记下读音,他的进步让人难以置信得快。他说汉语几乎没有语法,词汇知道了,他也就会说汉语了。
我还认识一个女孩,她在大学开始学汉语一年后她就开始给中国人做翻译了,学习两年的时候我看到她工作,她从说到写,都已经有不错的水平了。

下面是我读到过的一篇文章,也与大家共享: \sidenote{文中之文吗?}

汉语是最了不起的语言
记得的本人上中学的时候,语文课本上堂而皇之地写着:「走拼音化道路是汉语的必然趋势。」其中最主要的一条理由便是,英文可以打字,而汉语不能。现在回想起来真可笑。随著计算机技术的发展,汉字的键盘输入速度已远远超过英文,而且还在随着技术进步而不断快速提高。可英文呢?止步不前了吧。

现代所有学科领域,中国都有很好的学者,没听说哪位因汉语「不精确」而搞不好研究的。
中国的火箭照样可以精确升空,中国的原子弹照样可以精确爆炸。
所有的英文科技文献都可以翻译成汉语。
汉语文献影响力正随著国力的增强而在世界范围内增强。
下面举个最简单的例子来显示英文的笨拙:
本人曾问系里的几个教授「长方体cubic」 \sidenote{应为rectangular block} 如何用英文讲,
可这几位母语是英文的工科教授竟说不知道,
接下来连问几个本地的研究生,结果他们也不知道。
着实令我大吃一惊!现在我要问读者:您知道么?
反正不是 Cube,Rectangular...。
后来,我倒是真的在字典里找到了该词,可现在又忘了,原因是它太生僻。
感叹, 英文真是笨人的语言,试图给天下每一事物起一个名字。
宇宙无穷,英文词汇无穷!
词汇如「光幻觉」、「四环素」、「变阻器」、「碳酸钙」、「高血压」、「肾结石」、「七边形」、「五面体」
都只有专业人士才会。根本不可能像汉语那样触类旁通,不信?
去亲自问问母语是英文的人好了。
英文是发散的。搞的一些基本概念如「长方体」也只有专家才会讲!\sidenote{以偏盖全}

怪不得英文世界里专家那么多,而且都那么自信;是啊,一般人连他们的基本术语如「酒精绵球」「血压计」都不会讲。生活在英文世界真是对无知无奈!可悲可怜!英文是一维的,是密码语言。写英文是编码,读英文是解码。细想想:如把英文的a、b、c、d、e换成1、2、3、4、5,并没有甚么原则上的区别。按上边的对应,如一开始就把cab写成312没有甚么原则上的区别。
用一样的读音,又有甚么不可以?\sidenote{忽视词源}

汉语就不同了,是二维的(纸面上的最大维数),最大限度地利用了纸面的几何空间。
每个汉字就是一幅画。试问从一幅画上得到的信息快,还是从一行密码中得到的信息快?
国家汉字的扫盲标准是1500个字,理工科的大学生一般掌握2000个汉字。
就凭这2000个字,大家可以读书、看报、搞科研。
可在英文世界里,没有20000个\underline{字}\sidenote{词}别想读报,没有30000个字别想把《时代》周刊读顺,大学毕业10年后的职业人士一般都懂80000字。新事物的涌现,总伴随者英文新词,例如火箭(ROCKET),计算机(COMPUTER)等,可汉语则无须,不就是用「火」驱动的「箭」么,会「计算」的「机」么!可英文就不能这么干,不能靠组词,原因是「太长」了。如火箭将成为「FIRE-DRIVEN-ARROW」,计算机将成为「COMPUTAIONAL-MACHINE」等。人的视角有限,太长的字会降低文章的可读性与读者的理解能力。

目前,英文词汇已突破40万,预计下世纪中叶,将突破100万大关。
而汉语则相对稳定,现在中学生还可以琅琅上口地读屈原的\underline{楚词}\sidenote{古字不少,难读。}。
英文就难了,太不稳定。
现在的人们读沙士比亚的\underline{原著}\sidenote{余秋雨都能读懂。}已困难重重,更不用说读400年前英国诗人乔叟的诗了。
学GRE的时候,注意到很多韦氏字典收录的词汇竟是本世纪初的新词,如「Gargantuan」取自拉伯雷的小说。这也不奇怪,毕竟英文400年前才统一了拼写。
为汉语辩护!呼吁那些糟蹋汉语的人注意以下事实:
\begin{enumerate}[label=(\arabic*)]
\item 联合国6种文字的官方文件中最薄一本一定是汉语;
\item 汉语的精确性已为蓬勃发展的中国科技事业所证实;
\item 计算机语音输入最具有希望的是汉语;
\item 汉语是稳定的是收敛的,英文是不稳定的是发散的;
\item 汉语是二维信息是生动的是高效的,英文是一维信息是密码型的是枯躁低效的。
\item 在英文世界里能读文学名著是一件了不起的事,不是所有受过大学教育的人都能干的。
如阅读《荆棘鸟》中用英文描述的非州的一些植物真是艰涩无比,一般英美人也只能囫囵吞枣而已;
\footnote{如风读《The gay genius ,the life and times of Su Dongpo》丹药方面的内容时初有体会。但读译著同样不懂,不在哪种语言,是知识所限。}
可在中文世界里,又有谁会对仅有中学学历的人读完\underline{四大名著}\sidenote{囫囵吞枣而已}
而感到惊奇?
\end{enumerate}
不要给语言分高下。
\end{document}