\documentclass{ctexart}
\usepackage{xpinyin}
\IfFileExists{/dev/null}{
\IfFileExists{/Library/Fonts/Songti.ttc}{}
{\setCJKmainfont[BoldFont = NotoSansCJK-Bold.ttc]{NotoSansCJK-Regular.ttc}}
}{}
\usepackage{enumitem}
\usepackage[symbol]{footmisc}
\usepackage{sidenotes}
\title{春秋左氏傳/文公\footnote{Exported from Wikisource on 2021年4月6日}}
\begin{document}
\maketitle









\section{文公元年}





\textbf{經}



元年春,王正月,公即位。

二月,癸亥,日有食之。天王使叔服來會葬。

夏,四月,丁巳,葬我君僖公。天王使毛伯來錫公命。晉侯伐衛。叔孫得臣如京師。衛人伐晉。

秋,公孫敖會晉侯于戚。

冬,十月,丁未,楚世子商臣弒其君頵 \sidenote{楚成王,一作惲}。公孫敖如齊。

\textbf{傳}



元年春,王使內史叔服來會葬。公孫敖聞其能相人也,見其二子焉。叔服曰:「穀也食子,難也收子。穀也豐下,必有後於魯國。」

於是閏三月,非禮也 \sidenote{春秋时鲁国历法与周历基本相同,经传用鲁历,但传在记述晋国历史的有些部分用夏历}。
先王之正時也,履端於始,舉正於中,歸餘於終。履端於始,序則不愆。舉正於中,民則不惑。歸餘於終,事則不悖。

夏,四月,丁巳,葬僖公。

王使毛伯衛來錫公命,叔孫得臣如周拜。

晉文公之季年,諸侯朝晉。
衛成公不朝,使孔達侵鄭,伐緜、訾、及匡。
晉襄公既祥,使告于諸侯而伐衛,及南陽。先且居曰:「效尤,禍也。請君朝王,臣從師。」
晉侯朝王于溫,先且居、胥臣伐衛。

五月,辛酉朔,晉師圍戚。

六月,戊戌,取之,獲孫昭子。

衛人使告于陳,陳共公曰:「更伐之,我辭之。」衛孔達帥師伐晉。君子以為古,古者越國而謀。

秋,晉侯疆戚田,故公孫敖會之。

初,楚子將以商臣為大子,訪諸令尹子上,
子上曰:「君之齒未也,而又多愛,黜乃亂也。楚國之舉,恒在少者,
且是人也,蜂目而豺聲,忍人也,不可立也。」弗聽。
既又欲立王子職而黜大子商臣。商臣聞之而未察,告其師潘崇曰:「若之何而察之?」
潘崇曰:「享江芊而勿敬也。」從之。
江芊怒曰:「呼,役夫!宜君王之欲殺女而立職也。」告潘崇曰:「信矣。」
潘崇曰:「能事諸乎?」曰:「不能。」「能行乎?」曰:「不能。」「能行大事乎?」曰:「能。」

冬,十月,以宮甲圍成王。王請食熊蹯而死,弗聽。丁未,王縊。謚之曰「靈」,不瞑,曰「成」,乃瞑。

穆王立,以其為大子之室與潘崇,使為大師,且掌環列之尹。

穆伯如齊,始聘焉,禮也。
凡君即位,卿出並聘,踐脩舊好,要結外援,好事鄰國,以衛社稷,忠信卑讓之道也。
忠,德之正也。信,德之固也。卑讓,德之基也。

殽之役,晉人既歸秦師,秦大夫及左右皆言於秦伯曰:「是敗也,孟明之罪也,必殺之。」秦伯曰:「是孤之罪也。周芮良夫之詩曰:『大風有隧,貪人敗類。聽言則對,誦言如醉。匪用其良,覆俾我悖。』是貪故也,孤之謂矣。
孤實貪以禍夫子,夫子何罪?」復使為政。





\section{文公二年}


經



二年春,王二月,甲子,晉侯及秦師戰于彭衙,秦師敗績。

丁丑,作僖公主。

三月,乙巳,及晉處父盟。

夏,六月,公孫敖會宋公、陳侯、鄭伯、晉士縠盟于垂隴。

自十有二月不雨,至于秋七月。

八月,丁卯,大事于大廟,躋僖公。

冬,晉人、宋人、陳人、鄭人、伐秦。公子遂如齊納幣。

傳



二年春,秦孟明視帥師伐晉,以報殽之役。

二月,晉侯禦之。先且居將中軍、趙衰佐之;王官無地御戎;狐鞫居為右。
甲子,及秦師戰于彭衙,秦師敗績。晉人謂秦拜賜之師。戰于殽也,晉梁弘御戎、萊駒為右。
戰之明日,晉襄公縛秦囚,使萊駒以戈斬之。囚呼、萊駒失戈,狼\xpinyin{瞫}{shen3}取戈以斬囚,禽之以從公乘,遂以為右。箕之役,先軫黜之,而立續簡伯。狼瞫怒,其友曰:盍死之?瞫曰:吾未獲死所。其友曰:吾與女為難。瞫曰:《周志》有之:勇則害上,不登於明堂。死而不義,非勇也。共用之謂勇,吾以勇求右,無勇而黜,亦其所也。謂上不我知,黜而宜,乃知我矣。子姑待之!及彭衙既陳,以其屬馳秦師,死焉。晉師從之,大敗秦師。君子謂狼瞫於是乎君子。《詩》曰:君子如怒,亂庶遄沮。又曰:王赫斯怒,爰整其旅。怒不作亂,而以從師,可謂君子矣。秦伯猶用孟明。孟明增脩國政,重施於民。
\underline{趙成子}\sidenote{赵衰}言於諸大夫曰:「秦師又至,將必辟之。懼而增德,不可當也。《詩》曰:毋念爾祖,聿脩厥德。孟明念之矣。念德不怠,其可敵乎?」

丁丑,作僖公主。書,不時也。

晉人以公不朝,來討。公如晉。

夏四月,己巳,晉人使陽處父盟公,以恥之。書曰:及晉處父盟,以厭之也。適晉不書,諱之也。

公未至,六月,穆伯會諸侯,及晉司空士縠盟于垂隴,晉討衛故也。書士縠,堪其事也。陳侯為衛請成于晉,執孔達以說。

秋八月,丁卯,大事于大廟,躋僖公,逆祀也。於是夏父弗忌為宗伯,尊僖公,且明見曰:吾見新鬼大,故鬼小,先大後小,順也。躋聖賢,明也。明順,禮也。君子以為失禮。禮無不順。祀,國之大事也,而逆之,可謂禮乎?子雖齊聖,不先父食,久矣。故禹不先鯀,湯不先契,文武不先不窋。宋祖帝乙,鄭祖厲王,猶上祖也。是以《魯頌》曰:春秋匪解,享祀不忒。皇皇后帝,皇祖后稷。君子曰禮,謂其后稷親,而先帝也。詩曰:問我諸姑,遂及伯姊。君子曰:禮謂其姊親而先姑也。仲尼曰:臧文仲其不仁者三,不知者三。下展禽、廢六關、妾織蒲,三不仁也。作虛器、縱逆祀、祀爰居,三不知也。

冬,晉先且居、宋公子成、陳轅選、鄭公子歸生,伐秦,取汪及彭衙而還,以報彭衙之役。卿不書,為穆公故,尊秦也,謂之崇德。襄仲如齊納幣,禮也。凡君即位,好舅甥,脩昏姻,娶元妃以奉粢盛,孝也。孝,禮之始也。





文公三年


經



春,王正月,叔孫得臣會晉人、宋人、陳人、衛人、鄭人,伐沈,沈潰。

夏,五月,王子虎卒。秦人伐晉。

秋,楚人圍江,雨螽于宋。

冬,公如晉。

十有二月,己巳,公及晉侯盟。晉陽處父帥師伐楚,以救江。

傳



春,莊叔會諸侯之師伐沈,以其服於楚也。沈潰。凡民,逃其上曰潰,在上曰逃。衛侯如陳,拜晉成也。

夏四月,乙亥,王叔文公卒,來赴,弔如同盟,禮也。秦伯伐晉,濟河焚舟,取王官及郊。晉人不出,遂自茅津濟,封殽尸而還。遂霸西戎,用孟明也。君子是以知,秦穆公之為君也,舉人之周也,與人之壹也;孟明之臣也,其不解也,能懼思也;子桑之忠也,其知人也,能舉善也。《詩》曰:于以采蘩?于沼于沚。于以用之?公侯之事。秦穆有焉;夙夜匪解,以事一人。孟明有焉;詒厥孫謀,以燕翼子。子桑有焉。

秋,雨螽于宋,隊而死也。

楚師圍江,晉先僕伐楚以救江。

冬,晉以江故告于周。王叔桓公、晉陽處父,伐楚,以救江。門于方城,遇息公子朱而還。

晉人懼其無禮於公也,請改盟。公如晉,及晉侯盟。晉侯饗公,賦《菁菁者莪》,莊叔以公降拜,曰:小國受命於大國,敢不慎儀?君貺之以大禮,何樂如之?抑小國之樂,大國之惠也。晉侯降辭,登成拜,公賦《嘉樂》。





文公四年


經



春,公至自晉。

夏,逆婦姜于齊。狄侵齊。

秋,楚人滅江。晉侯伐秦,衛侯使甯俞來聘。

冬十有一月,壬寅,夫人風氏薨。

傳



春,晉人歸孔達于衛。以為衛之良也,故免之。

夏,衛侯如晉拜。曹伯如晉會正。逆婦姜于齊,卿不行,非禮也。君子是以知出姜之不允於魯也,曰:貴聘而賤逆之,君而卑之,立而廢之,棄信而壞其主,在國必亂,在家必亡。不允宜哉。《詩》曰:畏天之威,于時保之,敬主之謂也。

秋,晉侯伐秦,圍刓新城,以報王官之役,楚人滅江,秦伯為之降服,出次,不舉,過數,大夫諫公曰,同盟滅,雖不能救,敢不矜乎,吾自懼也,君子曰,詩云,惟彼二國,其政不獲,惟此四國,爰究爰度,其秦穆之謂矣,衛甯武子來聘,公與之宴,為賦湛露,及彤弓,不辭,又不荅賦,使行人私焉,對曰,臣以為肄業及之也,昔諸侯朝正於王,王宴樂之,於是乎賦湛露,則天子當陽,諸侯用命也,諸侯敵王所愾,而獻其功,王於是乎賜之,彤弓一,彤矢百,玈弓矢千,以覺報宴,今陪臣來繼舊好,君辱貺之,其敢干大禮以自取戾,

冬,成風薨。





文公五年


經



春,王正月,王使榮叔歸含,且賵。

三月,辛亥,葬我小君成風。王使召伯來會葬。

夏,公孫敖如晉。秦人入鄀。

秋,楚人滅六。

冬十月,甲申,許男業卒。

傳



春,王使榮叔來含,且賵。召昭公來會葬,禮也。初,鄀叛楚即秦,又貳於楚。

夏,秦人入鄀。六人叛楚,即東夷。

秋,楚成大心、仲歸帥師滅六。冬,楚子燮滅蓼。臧文仲聞六與蓼滅,曰:皋陶、庭堅不祀忽諸?德之不建,民之無援,哀哉!晉陽處父聘于衛,反過甯,甯嬴從之,及溫而還。其妻問之,嬴曰:以剛。《商書》曰:沈漸剛克,高明柔克。夫子壹之,其不沒乎!?天為剛德,猶不于時,況在人乎!且華而不實,怨之所聚也。犯而聚怨,不可以定身。余懼不獲其利,而離其難,是以去之。晉趙成子、欒貞子、霍伯、臼季,皆卒。





文公六年


經



春,葬許僖公。

夏,季孫行父如陳。秋,季孫行父如晉。

八月,乙亥,晉侯驩卒。

冬十月,公子遂如晉。葬晉襄公。晉殺其大夫陽處父。晉狐射姑出奔狄。

閏月不告月。猶朝于廟。

傳



春,晉蒐于夷,舍二軍。使狐射姑將中軍,趙盾佐之。陽處父至自溫,改蒐于董,易中軍。陽子,成季之屬也,故黨於趙氏,且謂趙盾能,曰:使能,國之利也。是以上之。宣子於是乎始為國政,制事典、正法罪、辟刑獄、董逋逃、由質要、治舊洿、本秩禮、續常職、出滯淹。既成,以授大傅陽子,與大師賈佗,使行諸晉國,以為常法。臧文仲以陳、衛之睦也,欲求好於陳。

夏,季文子聘于陳,且娶焉。秦伯任好卒,以子車氏之三子,奄息、仲行、鍼虎,為殉,皆秦之良也。國人哀之,為之賦《黃鳥》。君子曰:秦穆之不為盟主也,宜哉!死而棄民。先王違世,猶詒之法,而況奪之善人乎!?《詩》曰:人之云亡,邦國殄瘁。無善人之謂。若之何奪之?古之王者,知命之不長,是以並建聖哲、樹之風聲、分之采物、著之話言、為之律度、陳之藝極、引之表儀、予之法制、告之訓典、教之防利、委之常秩、道之以禮,則使毋失其土宜,眾隸賴之,而後即命。聖王同之。今縱無法以遺後嗣,而又收其良以死,難以在上矣。君子是以知秦之不復東征也。

秋,季文子將聘於晉,使求遭喪之禮以行。其人曰:將焉用之?文子曰:備豫不虞,古之善教也。求而無之,實難。過求何害?

八月,乙亥,晉襄公卒。靈公少,晉人以難故,欲立長君。趙孟曰:立公子雍。好善而長,先君愛之,且近於秦。秦舊好也。置善則固,事長則順,立愛則孝,結舊則安。為難故,故欲立長君。有此四德者,難必抒矣。賈季曰:不如立公子樂。辰嬴嬖於二君,立其子,民必安之。趙孟曰:辰嬴賤,班在九人,其子何震之有?且為二嬖,淫也。為先君子,不能求大,而出在小國,辟也。母淫子辟,無威;陳小而遠,無援。將何安焉?杜祁以君故,讓偪姞而上之;以狄故,讓季隗而已次之,故班在四。先君是以愛其子,而仕諸秦,為亞卿焉。秦大而近,足以為援;母義子愛,足以威民。立之,不亦可乎?使先蔑、士會,如秦,逆公子雍。賈季亦使召公子樂于陳。趙孟使殺諸郫。賈季怨陽子之易其班也,而知其無援於晉也。

九月,賈季使續鞫居殺陽處父。書曰晉殺其大夫,侵官也。

冬十月,襄仲如晉。葬襄公。

十一月,丙寅,晉殺續簡伯。賈季奔狄。宣子使臾駢送其帑。夷之蒐,賈季戮臾駢。臾駢之人,欲盡殺賈氏以報焉。臾駢曰:不可。吾聞前志有之曰:敵惠敵怨,不在後嗣,忠之道也。夫子禮於賈季,我以其寵報私怨,無乃不可乎!?介人之寵,非勇也;損怨益仇,非知也;以私害公,非忠也。釋此三者,何以事夫子?!盡具其帑,與其器用財賄,親師扞之,送致諸竟。閏月不告朔,非禮也。閏以正時,時以作事,事以厚生,生民之道,於是乎在矣。不告閏朔,棄時政也,何以為民?





文公七年


經



春,公伐邾,

三月,甲戌,取須句,遂城郚,

夏,四月,宋公王臣卒,宋人殺其大夫,戊子,晉人及秦人戰于令狐,晉先蔑奔秦,狄侵我西鄙,

秋,八月,公會諸侯,晉大夫,盟于扈,

冬,徐伐莒,公孫敖如莒蒞盟,

傳



春,公伐邾,間晉難也,

三月,甲戌,取須句,寘文公子焉,非禮也,

夏,四月,宋成公卒,於是公子成為右師,公孫友為左師,樂豫為司馬,鱗矔為司徒,公子蕩為司城,華御事為司寇,昭公將去群公子,樂豫曰,不可,公族,公室之枝葉也,若去之,則本根無所庇陰矣,葛藟猶能庇其本根,故君子以為比,況國君乎,此諺所謂庇焉,而縱尋斧焉者也,必不可,君其圖之,親之以德,皆股肱也,誰敢攜貳,若之何去之,不聽,穆襄之族,率國人以攻公,殺公孫固,公孫鄭,于公宮,六卿和公室,樂豫舍司馬,以讓公子卬,昭公即位而葬,書曰,宋人殺其大夫,不稱名,眾也,且言非其罪也,秦康公送公子雍于晉,曰,文公之入也,無衛,故有呂郤之難,乃多與之徒衛,穆嬴日抱太子以啼于朝,曰,先君何罪,其嗣亦何罪,舍適嗣不立,而外求君,將焉寘此,出朝則抱以適趙氏,頓首於宣子曰,先君奉此子也,而屬諸子曰,此子也才,吾受子之賜,不才,吾唯子之怨,今君雖終,言猶在耳,而棄之,若何,宣子與諸大夫皆患穆嬴,且畏偪,乃背先蔑而立靈公,以禦秦師,箕鄭居守,趙盾將中軍,先克佐之,荀林父佐上軍,先蔑將下軍,先都佐之,步招御戎,戎津為右,及堇陰,宣子曰,我若受秦,秦則賓也,不受,寇也,既不受矣,而復緩師,秦將生心,先人有奪人之心,軍之善謀也,逐寇如追逃,軍之善政也,訓卒利兵,秣馬蓐食,潛師夜起,戊子,敗秦師于令狐,至于刳首,己丑,先蔑奔秦,士會從之,先蔑之使也,荀林父止之曰,夫人大子猶在,而外求君,此必不行,子以疾辭,若何,不然,將及,攝卿以往,可也,何必子,同官為寮,吾嘗同寮,敢不盡心乎,弗聽,為賦板之三章,又弗聽,及亡,荀伯盡送其帑,及其器用財賄於秦,曰,為同寮故也,士會在秦三年,不見士伯,其人曰,能亡人於國,不能見於此,焉用之,士季曰,吾與之同罪,非義之也,將何見焉,及歸,遂不見,狄侵我西鄙,公使告于晉,趙宣子使因賈季問酆舒,且讓之,酆舒問於賈季,曰,趙衰,趙盾,孰賢,對曰,趙衰,冬日之日也,趙盾,夏日之日也,

秋,八月,齊侯,宋公,衛侯,鄭伯,許男,曹伯,會晉趙盾,盟于扈,晉侯立故也,公後至,故不書所會,凡會諸侯,不書所會,後也,後至不書其國,辟不敏也,穆伯娶于莒,曰,戴已,生文伯,其娣聲已,生惠叔,戴已卒,又聘于莒,莒人以聲已辭,則為襄仲聘焉,

冬,徐伐莒,莒人來請盟,穆伯如莒蒞盟,且為仲逆,及鄢陵,登城見之,美,自為娶之,仲請攻之,公將許之,叔仲惠伯諫曰,臣聞之,兵作於內為亂,於外為寇,寇猶及人,亂自及也,今臣作亂,而君不禁,以啟寇讎,若之何,公止之,惠伯成之,使仲舍之,公孫敖反之,復為兄弟如初,從之,晉郤缺言於趙宣子曰,日衛不睦,故取其地,今已睦矣,可以歸之,叛而不討,何以示威,服而不柔,何以示懷,非威非懷,何以示德,無德,何以主盟,子為正卿,以主諸侯,而不務德,將若之何,夏書曰,戒之用休,董之用威,勸之以九歌,勿使壞,九功之德,皆可歌也,謂之九歌,六府三事,謂之九功,水,火,金,木,土,穀,謂之六府,正德,利用,厚生,謂之三事,義而行之,謂之德禮,無禮不樂,所由叛也,若吾子之德,莫可歌也,其誰來之,盍使睦者歌吾子乎,宣子說之,





文公八年


經



春,王正月,

夏,四月,

秋,八月,戊申,天王崩,

冬,十月,壬午,公子遂會晉趙盾,盟于衡雍,乙酉,公子遂會雒戎,盟于暴,公孫敖如京師,不至而復,丙戌,奔莒,螽,宋人殺其大夫司馬,宋司城來奔,

傳



春,晉侯使解揚歸匡戚之田于衛,且復致公婿池之封,自申至于虎牢之竟,

夏,秦人伐晉,取武城,以報令狐之役,

秋,襄王崩,晉人以扈之盟來討,

冬,襄仲會晉趙孟,盟于衡雍,報扈之盟也,遂會伊雒之戎,書曰,公子遂,珍之也,穆伯如周弔喪,不至,以幣奔莒,從已氏焉,宋襄夫人,襄王之姊也,昭公不禮焉,夫人因戴氏之族,以殺襄公之孫孔叔,公孫鍾離,及大司馬公子卬,皆昭公之黨也,司馬握節以死,故書以官,司城蕩意諸來奔,效節於府人而出,公以其官逆之,皆復之,亦書以官,皆貴之也,夷之蒐,晉侯將登箕鄭父,先都,而使士縠,梁益耳,將中軍,先克曰,狐趙之勳,不可廢也,從之,先克奪蒯得田于堇陰,故箕鄭父,先都,士縠,梁益耳,蒯得,作亂,





文公九年


經



春,毛伯來求金,夫人姜氏如齊,

二月,叔孫得臣如京師,辛丑,葬襄王,晉人殺其大夫先都,

三月,夫人姜氏至自齊,晉人殺其大夫士縠,及箕鄭父,楚人伐鄭,公子遂會晉人,宋人,衛人,許人,救鄭,

夏,狄侵齊,

秋,八月,曹伯襄卒,

九月,癸酉,地震,

冬,楚子使椒來聘,秦人來歸僖公成風之隧,葬曹共公,

傳



春,王正月,己酉,使賊殺先克,乙丑,晉人殺先都,梁益耳,毛伯衛來求金,非禮也,不書王命,未葬也,

二月,莊叔如周葬襄王,

三月,甲戌,晉人殺箕鄭父,士縠,蒯得,范山言於楚子曰,晉君少,不在諸侯,北方可圖也,楚子師于狼淵以伐鄭,囚公子堅,公子尨,及樂耳,鄭及楚平,公子遂會晉趙盾,宋華耦,衛孔達,許大夫,救鄭,不及楚師,卿不書,緩也,以懲不恪,

夏,楚侵陳,克壺丘,以其服於晉也,

秋,楚公子朱自東夷伐陳,陳人敗之,獲公子茷,陳懼,乃及楚平,

冬,楚子越椒來聘,執幣傲,叔仲惠伯曰,是必滅若敖氏之宗,傲其先君,神弗福也,秦人來歸僖公成風之襚,禮也,諸侯相弔賀也,雖不當事,苟有禮焉 ,書也,以無忘舊好,





文公十年


經



春,王三月,辛卯,臧孫辰卒,

夏,秦伐晉,楚殺其大夫宜申,自正月不雨,至于秋七月,及蘇子盟于女栗,

冬,狄侵宋,楚子,蔡侯,次于厥貉,

傳



春,晉人伐秦,取少梁,

夏,秦伯伐晉,取北徵,初楚范巫矞似,謂成王,與子玉,子西,曰,三君皆將強死,城濮之役,王思之,故使止子玉曰,毋死,不及,止子西,子西縊而縣絕,王使適至,遂止之,使為商公,沿漢泝江,將入郢,王在渚宮,下見之,懼而辭曰,臣免於死,又有讒言,謂臣將逃,臣歸死於司敗也,王使為工尹,又與子家謀弒穆王,穆王聞之,

五月,殺鬥宜申,及仲歸,

秋,七月,及蘇子盟于女栗,頃王立故也,陳侯,鄭伯,會楚子于息,

冬,遂及蔡侯,次于厥貉,將以伐宋,宋華御事曰,楚欲弱我也,先為之弱乎,何必使誘我,我實不能,民何罪,乃逆楚子,勞且聽命,遂,道以田孟諸,宋公為右盂,鄭伯為左盂,期思公復遂,為右司馬,子朱及文之無畏,為左司馬,命夙駕載燧,宋公違命,無畏抶其僕以徇,或謂子舟曰,國君不可戮也,子舟曰,當官而行,何彊之有,詩曰,剛亦不吐,柔亦不茹,毋縱詭隨,以謹罔極,是亦非辟彊也,敢愛死以亂官乎,厥貉之會,麇子逃歸,





文公十一年


經



春,楚子伐麇,

夏,叔仲彭生,會晉郤缺于承筐,

秋,曹伯來朝,公子遂如宋,狄侵齊,

冬,十月,甲午,叔孫得臣敗狄于鹹,

傳



春,楚子伐麇,成大心敗麇師於防渚,潘崇復伐麇,至于鍚穴,

夏,叔仲惠伯會晉郤缺于承筐,謀諸侯之從於楚者,

秋,曹文公來朝,即位而來見也,襄仲聘于宋,且言司城蕩意諸而復之,因賀楚師之不害也,鄋瞞侵齊,遂伐我,公卜使叔孫得臣追之,吉,侯叔夏御莊叔,綿房甥為右,富父終甥駟乘,

冬,十月,甲午,敗狄于鹹,獲長狄僑如,富父終甥樁其喉,以戈殺之,埋其首於子駒之門,以命宣伯,初,宋武公之世,鄋瞞伐宋,司徒皇父帥師禦之,耏班御皇父充石,公子穀甥為右,司寇牛父駟乘,以敗狄于長丘,獲長狄緣斯,皇父之二子死焉,宋公於是以門賞耏班,使食其征謂之耏門,晉之滅潞也,獲僑如之弟焚如,齊襄公之二年,鄋瞞伐齊,齊王子成父獲其弟榮如,埋其首於周首之北門,衛人獲其季弟簡如,鄋瞞由是遂亡,郕大子朱儒,自安於夫鍾,國人弗徇,





文公十二年


經



春,王正月,郕伯來奔,杞伯來朝,

二月,庚子,子叔姬卒,

夏,楚人圍巢,

秋,滕子來朝,秦伯使術來聘,

冬,十有二月,戊午,晉人,秦人,戰于河曲,季孫行父帥師,城諸及鄆,

傳



春,郕伯卒,郕人立君,大子以夫鍾與郕邽來奔,公以諸侯逆之,非禮也,故書曰,郕伯來奔,不書地,尊諸侯也,杞桓公來朝,始朝公也,且請絕叔姬,而無絕昏,公許之,

二月,叔姬卒,不言杞,絕也,書叔姬,言非女也,楚令尹大孫伯卒,成嘉為令尹,群舒叛楚,夏,子孔執舒子平,及宗子,遂圍巢,

秋,滕昭公來朝,亦始朝公也,秦伯使西乞術來聘,且言將伐晉,襄仲辭玉曰,君不忘先君之好,照臨魯國,鎮撫其社稷,重之以大器,寡君敢辭玉,對曰,不腆敝器,不足辭也,主人三辭,賓客曰,寡君願徼福于周公,魯公,以事君,不腆先君之敝器,使下臣致諸執事,以為瑞節,要結好命,所以藉寡君之命,結二國之好,是以敢致之,襄仲曰,不有君子,其能國乎,國無陋矣,厚賄之,秦為令狐之役故,

冬,秦伯伐晉,取羈馬,晉人禦之,趙盾將中軍,荀林父佐之,郤缺將上軍,臾駢佐之,欒盾將下軍,胥甲佐之,范無恤御戎,以從秦師于河曲,臾駢曰,秦不能久,請深壘固軍以待之,從之,秦人欲戰,秦伯謂士會曰,若何而戰,對曰,趙氏新出其屬曰臾駢,必實為此謀,將以老我師也,趙有側室曰穿,晉君之婿也,有寵而弱,不在軍事,好勇而狂,且惡臾駢之佐上軍也,若使輕者肆焉,其可,秦伯以璧祈戰于河,十二月,戊午,秦軍掩晉上軍,趙穿追之不及,反,怒曰,裹糧坐甲,固敵是求,敵至不擊,將何俟焉,軍吏曰,將有待也,穿曰我不知謀,將獨出,乃以其屬出,宣子曰,秦獲穿也,獲一卿矣,秦以勝歸,我何以報,乃皆出戰,交綏,秦行人夜戒晉師曰,兩君之士,皆未憖也,明日請相見也,臾駢曰,使者目動而言肆,懼我也,將遁矣,薄諸河,必敗之,胥甲,趙穿,當軍門呼曰,死傷未收而棄之,不惠也,不待期而薄人於險,無勇也,乃止,秦師夜遁,復侵晉,入瑕,城諸及鄆,書時也,





文公十三年


經



春,王正月,

夏,五月,壬午,陳侯朔卒,邾子蘧蒢卒,自正月不雨,至于秋七月,大室屋壞,

冬,公如晉,衛侯會公于沓,狄侵衛,

十有二月,己丑,公及晉侯盟,公還自晉,鄭伯會公于棐,

傳



春,晉侯使詹嘉處瑕,以守桃林之塞,晉人患秦之用士會也,

夏,六卿相見於諸浮,趙宣子曰,隨會在秦,賈季在狄,難日至矣,若之何,中行桓子曰,請復賈季,能外事,且由舊勳,郤成子曰,賈季亂,且罪大,不如隨會,能賤而有恥,柔而不犯,其知足使也,且無罪,乃使魏壽餘偽以魏叛者,以誘士會,執其帑於晉,使夜逸,請自歸于秦,秦伯許之,履士會之足於朝,秦伯師于河西,魏人在東,壽餘曰,請東人之能與夫二三有司言者,吾與之,先使士會,士會辭曰,晉人虎狼也,若背其言,臣死,妻子為戮,無益於君,不可悔也,秦伯曰,若背其言,所不歸爾帑者,有如河,乃行,繞朝贈之以策,曰子無謂秦無人,吾謀適不用也,既濟,魏人譟而還,秦人歸其帑,其處者為劉氏,邾文公卜遷于繹,史曰,利於民而不利於君,邾子曰,苟利於民,孤之利也,天生民而樹之君,以利之也,民既利矣,孤必與焉,左右曰,命可長也,君何弗為,邾子曰,命在養民,死之短長,時也,民苟利矣,遷也,吉莫如之,遂遷于繹,五月,邾文公卒,君子曰知命,

秋,七月,大室之屋壞,書不共也,

冬,公如晉朝且尋盟,衛侯會公于沓,請平于晉,公還,鄭伯會公于棐,亦請平于晉,公皆成之,鄭伯與公宴于棐,子家賦鴻鴈,季文子曰,寡君未免於此,文子賦四月,子家賦載馳之四章,文子賦采薇之四章,鄭伯拜,公荅拜,





文公十四年


經



春,王正月,公至自晉,邾人伐我南鄙,叔彭生帥帥伐邾,

夏,五月,乙亥,齊侯潘卒,

六月,公會宋公,陳侯,衛侯,鄭伯,許男,曹伯,晉趙盾,癸酉,同盟于新城,

秋,七月,有星孛入于北斗,公至自會,晉人納捷菑于邾,弗克納,

九月,甲申,公孫敖卒于齊,齊公子商人弒其君舍,宋子哀來奔,

冬,單伯如齊,齊人執單伯,齊人執子叔姬,

傳



春,頃王崩,周公閱與王孫,蘇爭政,故不赴,凡崩薨,不赴則不書,禍福不告亦不書,不懲敬也,邾文公之卒也,公使弔焉,不敬,邾人來討,伐我南鄙,故惠伯伐邾子叔姬妃齊昭公,生舍,叔姬無寵,舍無威,公子商人驟施於國,而多聚士,盡其家,貸於公有司以繼之,

夏,五月,昭公卒,舍即位,邾文公元妃齊姜,生定公,二妃晉姬,生捷菑,文公卒,邾人立定公,捷菑奔晉,

六月,同盟于新城,從於楚者服,且謀邾也,

秋,七月,乙卯夜,齊商人弒舍而讓元,元曰,爾求之久矣,我能事爾,爾不可使多蓄憾,將免我乎,爾為之,有星孛入于北斗,周內史叔服曰,不出七年,宋齊晉之君,皆將死亂,晉趙盾以諸侯之師八百乘,納捷菑于邾,邾人辭曰,齊出貜且長,宣子曰,辭順而弗從,不祥,乃還,周公將與王孫蘇訟于晉,王叛王孫蘇,而使尹氏與聃啟,訟周公于晉,趙宣子平王室而復之,楚莊王立,子孔,潘崇,將襲群舒,使公子燮與子儀守,而伐舒蓼,二子作亂,城郢,而使賊殺子孔,不克而還,

八月,二子以楚子出,將如商密,廬戢黎及叔麇誘之,遂殺鬥克,及公子燮,初,鬥克囚于秦,秦有殽之敗,而使歸求成,成而不得志,公子燮求令尹而不得,故二子作亂,穆伯之從己氏也,魯人立文伯,穆伯生二子於莒,而求復,文伯以為請,襄仲使無朝聽命,復而不出,二年而盡室以復適莒,文伯疾,而請曰,穀之子弱,請立難也,許之,文伯卒,立惠叔,穆伯請重賂以求復,惠叔以為請,許之,將來,

九月,卒于齊,告喪,請葬,弗許,宋高哀為蕭封人,以為卿,不義宋公而出,遂來奔,書曰,宋子哀來奔,貴之也,齊人定懿公,使來告難,故書以九月,齊公子元,不順懿公之為政也,終不曰公,曰夫已氏,襄仲使告于王,請以王寵,求昭姬于齊,曰,殺其子,焉用其母,請受而罪之,

冬,單伯如齊,請子叔姬,齊人執之,又執子叔姬,





文公十五年


經



春,季孫行父如晉,

三月,宋司馬華孫來盟,

夏,曹伯來朝,齊人歸公孫敖之喪,

六月,辛丑,朔,日有食之,鼓用牲于社,單伯至自齊,晉郤缺帥師伐蔡,戊申,入蔡,

秋,齊人侵我西鄙,季孫行父如晉,

冬,十有一月,諸侯盟于扈,

十有二月,齊人來歸子叔姬,齊侯侵我西鄙,遂伐曹,入其郛,

傳



春,季文子如晉,為單伯與子叔姬故也,

三月,宋華耦來盟,其官皆從之,書曰,宋司馬華孫,貴之也,公與之宴,辭曰,君之先臣督,得罪於宋殤公,名在諸侯之策,臣承其祀,其敢辱君,請承命於亞旅,魯人以為敏,

夏,曹伯來朝,禮也,諸侯五年再相朝,以脩王命,古之制也,齊人或為孟氏謀,曰,魯爾親也,飾棺寘諸堂阜,魯必取之,從之,卞人以告,惠叔猶毀以為請,立於朝以待命,許之,取而殯之,齊人送之書曰,齊人歸公孫敖之喪,為孟氏,且國故也,葬視共仲,聲己不視,帷堂而哭,襄仲欲勿哭,惠伯曰喪,親之終也,雖不能始,善終可也,史佚有言曰,兄弟致美,救乏,賀善,弔災,祭敬,喪哀,情雖不同,毋絕其愛親之道也,子無失道,何怨於人,襄仲說,帥兄弟以哭之,他年,其二子來,孟獻子愛之,聞於國,或譖之曰,將殺子,獻子以告,季文子二子曰,夫子以愛我聞,我以將殺子聞,不亦遠於禮乎,遠禮不如死,一人門于句鼆,一人門于戾丘,皆死,

六月,辛丑朔,日有食之,鼓用牲于社,非禮也,日有食之,天子不舉,伐鼓于社,諸侯用幣于社,伐鼓于朝,以昭事神,訓民事君,示有等威,古之道也,齊人許單伯請而赦之,使來致命,書曰,單伯至自齊,貴之也,新城之盟,蔡人不與,晉郤缺以上軍下軍伐蔡,曰,君弱,不可以怠,戊申,入蔡,以城下之盟而還,凡勝國,曰滅之,獲大城焉,曰入之,

秋,齊人侵我西鄙,故季文子告于晉,

冬,十一月,晉侯,宋公,衛侯,蔡侯,鄭伯許男,曹伯,盟于扈,尋新城之盟,且謀伐齊也,齊人賂晉侯,故不克而還,於是有齊難,是以公不會,書曰,諸侯盟于扈,無能為故也,凡諸侯會,公不與不書,諱君惡也,與而不書,後也,齊人來歸子叔姬,王故也,齊侯侵我西鄙,謂諸侯不能也,遂伐曹,人其郛,討其來朝也,季文子曰,齊侯其不免乎,己則無禮,而討於有禮者,曰,女何故行禮,禮以順天,天之道也,己則反天,而又以討人,難以免矣,詩曰,胡不相畏,不畏于天,君子之不虐幼賤,畏于天也,在周頌曰,畏天之威,于時保之,不畏于天,將何能保,以亂取國,奉禮以守,猶懼不終,多行無禮,弗能在矣,





文公十六年


經



春,季孫行父會齊侯于陽穀,齊侯弗及盟,

夏,五月,公四不視朔,

六月,戊辰,公子遂及齊侯,盟于郪丘,

秋,八月,辛未,夫人姜氏薨,毀泉臺,楚人,秦人,巴人,滅庸,

冬,十有一月,宋人弒其君杵臼,

傳



春,王正月,及齊平,公有疾,使季文子會齊侯于陽穀,請盟,齊侯不肯,曰,請俟君間,

夏,五月,公四不視朔,疾也,公使襄仲納賂于齊侯,故盟于郪丘,有蛇自泉宮出,入于國,如先君之數,

秋,八月,辛未,聲姜薨,毀泉臺,楚大饑,戎伐其西南,至于阜山,師于大林,又伐其東南,至于陽丘,以侵訾枝,庸人帥群蠻以叛楚,麇人率百濮聚於選,將伐楚,於是申息之北門不啟,楚人謀徙於阪高,蒍賈曰,不可,我能往,寇亦能往,不如伐庸,夫麇與百濮,謂我饑不能師,故伐我也,若我出師,必懼而歸,百濮離居,將各走其邑,誰暇謀人,乃出師,旬有五日,百濮乃罷,自廬以往,振廩同食,次于句澨,使廬戢黎侵庸,及庸方城,庸人逐之,囚子揚窗,三宿而逸,曰,庸師眾,群蠻聚焉,不如復大師,且起王卒,合而後進,師叔曰,不可,姑又與之遇,以驕之,彼驕我怒,而後可克,先君蚡冒所以服陘隰也,又與之遇,七遇皆北,唯裨,鯈,魚,人實逐之,庸人曰,楚不足與戰矣,遂不設備,楚子乘馹,會師于臨品,分為二隊,子越自石溪,子貝自仞,以伐庸,秦人巴人從楚師,群蠻從楚子盟,遂滅庸,宋公子鮑禮於國人,宋饑,竭其粟而貸之,年自七十以上,無不饋詒也,時加羞珍異,無日不數於六卿之門,國之材人,無不事也,親自桓以下,無不恤也,公子鮑美而豔,襄夫人欲通之,而不可,夫人助之施,昭公無道,國人奉公子鮑以因夫人,於是華元為右師,公孫友為左師,華耦為司馬,鱗鱹為司徒,蕩意諸為司城,公子朝為司寇,初,司城蕩卒,公孫壽辭司城,請使意諸為之,既而告人曰,君無道,吾官近,懼及焉,棄官則族無所庇,子,身之貳也,姑紓死焉,雖亡子,猶不亡族,既,夫人將使公田孟諸而殺之,公知之,盡以寶行,蕩意諸曰,盍適諸侯,公曰,不能其大夫,至于君祖母,以及國人,諸侯誰納我,且既為人君,而又為人臣,不如死,盡以其寶賜左右,以使行,夫人使謂司城去公,對曰,臣之而逃其難,若後君何,

冬,十一月,甲寅,宋昭公將田孟諸,未至,夫人王姬使帥甸攻而殺之,蕩意諸死之,書曰,宋人弒其君杵臼,君無道也,文公即位,使母弟須為司城,華耦卒,而使蕩虺為司馬,





文公十七年


經



春,晉人,衛人,陳人,鄭人,伐宋,

夏,四月,癸亥,葬我小君聲姜,齊侯伐我西鄙,

六月,癸未,公及齊侯盟于穀,諸侯會于扈,

秋,公至自穀,

冬,公子遂如齊,

傳



春,晉荀林父,衛孔達,陳公孫寧,鄭石楚,伐宋,討曰,何故弒君,猶立文公而還,卿不書,失其所也,

夏,四月,癸亥,葬聲姜,有齊難,是以緩,齊侯伐我北鄙,襄仲請盟,

六月,盟于穀,晉侯蒐于黃父,遂復合諸侯于扈,平宋也,公不與會,齊難故也,書曰諸侯,無功也,於是晉侯不見鄭伯,以為貳於楚也,鄭子家使執訊而與之書,以告趙宣子,曰,寡君即位三年,召蔡侯而與之事君,九月,蔡侯入于敝邑以行,敝邑以侯宣多之難,寡君是以不得與蔡侯偕,

十一月,克減侯宣多,而隨蔡侯以朝于執事,

十二年,六月,歸生佐寡君之嫡夷,以請陳侯于楚,而朝諸君,

十四年,七月,寡君又朝,以蕆陳事,

十五年,五月,陳侯自敝邑往朝于君,往年正月,燭之武往朝夷也,八月,寡君又往朝,以陳蔡之密邇於楚,而不敢貳焉,則敝邑之故也,雖敝邑之事君,何以不免,在位之中,一朝于襄,而再見于君,夷與孤之二三臣,相及於絳,雖我小國,則篾以過之矣,今大國曰,爾未逞吾志,敝邑有亡,無以加焉,古人有言曰,畏首畏尾,身其餘幾,又曰,鹿死不擇音,小國之事大國也,德,則其人也,不德,則其鹿也,鋌而走險,急何能擇,命之罔極,亦知亡矣,將悉敝賦,以待於鯈,唯執事命之,

文公二年,六月,壬申,朝于齊,

四年,二月,壬戌,為齊侵蔡,亦獲成於楚,居大國之間,而從於強令,豈其罪也,大國若弗圖,無所逃命,晉鞏朔行成於鄭,趙穿,公婿池,為質焉,

秋,周甘歜敗戎于邥垂,乘其飲酒也,

冬,十月,鄭大子夷,石楚,為質于晉,襄仲如齊,拜穀之盟,復曰,臣聞齊人將食魯之麥,以臣觀之,將不能齊君之語偷,臧文仲有言曰,民主偷必死,





文公十八年


經



春,王二月,丁丑,公薨于臺下,秦伯罃卒,

夏,五月,戊戌,齊人弒其君商人,

六月,癸酉,葬我君文公,

秋,公子遂,叔孫得臣,如齊,

冬,十月,子卒,夫人姜氏歸于齊,季孫行父如齊,莒弒其君庶其,

傳



春,齊侯戒師期,而有疾,醫曰,不及秋,將死,公聞之,卜曰,尚無及期,惠伯令龜,卜楚丘占之,曰,齊侯不及期,非疾也,君亦不聞,令龜有咎,二月,丁丑,公薨,

齊懿公之為公子也,與邴歜之父爭田,弗勝,及即位,乃掘而刖之,而使歜僕,納閻職之妻,而使職驂乘,

夏,五月,公游于申池,二人浴于池,歜以扑抶職,職怒,歜曰,人奪女妻而不怒,一抶女庸何傷,職曰,與刖其父而弗能病者何如,乃謀弒懿公,納諸竹中,歸舍爵而行,齊人立公子元,

六月,葬文公,

秋,襄仲,莊叔,如齊,惠公立故,且拜葬也,文公二妃,敬嬴生宣公,敬嬴嬖,而私事襄仲,宣公長,而屬諸襄仲,襄仲欲立之,叔仲不可,仲見于齊侯而請之,齊侯新立,而欲親魯,許之,

冬,十月,仲殺惡及視,而立宣公,書曰,子卒,諱之也,仲以君命召惠伯,其宰公冉務人止之,曰,入必死,叔仲曰,死君命可也,公冉務人曰,若君命可死,非君命何聽,弗聽,乃入,殺而埋之馬矢之中,公冉務人奉其帑以奔蔡,既而復叔仲氏,

夫人姜氏歸于齊,大歸也,將行哭而過市,曰天乎,仲為不道,殺適,立庶,市人皆哭,魯人謂之哀姜,

莒紀公子生大子僕,又生季佗,愛季佗而黜僕,且多行禮於國,僕因國人以弒紀公,以其寶玉來奔,納諸宣公,公命與之邑,曰,今日必授,季文子使司寇出諸竟,曰,今日必達,公問其故,季文子使大史克對曰,先大夫臧文仲,教行父事君之禮,行父奉以周旋,弗敢失隊,曰,見有禮於其君者事之,如孝子之養父母也,見無禮於其君者誅之,如鷹鸇之逐鳥雀也,先君周公制周禮曰,則以觀德,德以處事,事以度功,功以食民,作誓命曰,毀則為賊,掩賊為藏,竊賄為盜,盜器為姦,主藏之名,賴姦之用,為大凶德,有常無赦,在九刑不忘,行父還觀莒僕,莫可則也,孝敬忠信為吉德,盜賊藏姦為凶德,夫莒僕,則其孝敬,則弒君父矣,則其忠信,則竊寶玉矣,其人,則盜賊也,其器,則姦兆也,保而利之,則主藏也,以訓則昏,民無則焉,不度於善,而皆在於凶德,是以去之,

昔高陽氏有才子八人,蒼舒,隤敳,檮戭,大臨,尨降,庭堅,仲容,叔達,齊聖廣淵,明允篤誠,天下之民,謂之八愷,高辛氏有才子八人,伯奮,仲堪,叔獻,季仲,伯虎,仲熊,叔豹,季貍,忠肅共懿,宣慈惠和,天下之民,謂之八元,此十六族也,世濟其美,不隕其名,以至於堯,堯不能舉,舜臣堯,舉八愷,使主后土,以揆百事,莫不時序,地平天成,舉八元,使布五教于四方,父義,母慈,兄友,弟共,子孝,內平,外成,

昔帝鴻氏有不才子,掩義隱賊,好行凶德,醜類惡物,頑嚚不友,是與比周,天下之民,謂之渾敦,少皞氏有不才子,毀信廢忠,崇飾惡言,靖譖庸回,服讒蒐慝,以誣盛德,天下之民,謂之窮奇,顓頊有不才子,不可教訓,不知話言,告之則頑,舍之則嚚,傲很明德,以亂天常,天下之民,謂之檮杌,此三族也,世濟其凶,增其惡名,以至于堯,堯不能去,縉雲氏有不才子,貪于飲食,冒于貨賄,侵欲崇侈,不可盈厭,聚斂積實,不知紀極,不分孤寡,不恤窮匱,天下之民,以比三凶,謂之饕餮,舜臣堯,賓于四門,流四凶族,渾敦,窮奇,檮杌,饕餮,投諸四裔,以禦螭魅,是以堯崩而天下如一,同心戴舜,以為天子,以其舉十六相,去四凶也,故虞書數舜之功曰,慎徽五典,五典克從,無違教也,曰,納于百揆,百揆時序,無廢事也,曰賓于四門,四門穆穆,無凶人也,舜有大功二十而為天子,今行父雖未獲一吉人,去一凶矣,於舜之功,二十之一也,庶幾免於戾乎,

宋武氏之族道昭公子,將奉司城須以作亂,十二月,宋公殺母弟須,及昭公子使戴,莊,桓,之族,攻武氏於司馬子伯之館,遂出武穆之族,使公孫師為司城,公子朝卒,使樂呂為司寇,以靖國人,
\end{document}
